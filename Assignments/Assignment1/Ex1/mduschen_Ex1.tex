\documentclass[12pt,letterpaper]{article}


\newcommand{\doctitle}{Scientific Computing Research Summary}
\newcommand{\docauthor}{Matthew Duschenes}
\newcommand{\docaffil}{Department of Applied Physics, University of Michigan}
\newcommand{\docheader}{NERS 570 \ - Homework 1 - \docauthor}
\newcommand{\docfooter}{\docaffil}


% Margins
\usepackage[margin=0.75in,marginparsep=0pt,paperwidth=216mm,paperheight=279mm]{geometry}

% Headers
\usepackage{fancyhdr}
\geometry{headheight=15pt}
\renewcommand{\headrulewidth}{0.4pt}% default is 0.4pt
\renewcommand{\footrulewidth}{0.4pt}% default is 0pt
\geometry{headheight=15pt}
\geometry{headsep=10pt}
\setlength{\skip\footins}{10pt} % gap between text and footer
\fancyhf{}
\fancyhead[R]{\docheader}
\fancyfoot[LE,RO]{\thepage}
\fancyfoot[LO,RE]{\docfooter}

% \makeatletter
% \if@twoside
% \fancyfoot[LE,RO]{\thepage}
% \fancyfoot[LO,RE]{\docfooter}
% \else
% \fancyfoot[R]{\thepage}
% \fancyfoot[L]{\docfooter}
% \fi
% \makeatother
% \fancyhead[R]{\docheader}


% Title
\usepackage{titling}
% \usepackage[affil-it]{authblk}
\usepackage[nodayofweek]{datetime}
\usepackage[super]{nth}

% \newdateformat{monthdayyear}{%
%   \monthname[\THEMONTH]~\THEDAY,~ \THEYEAR}
% \newdateformat{mydate}{\monthname[\THEMONTH] \nth[\THEDAY], \THEYEAR}


% Make Title
\pagestyle{fancy}
% \renewcommand*{\Authfont}{\bfseries}
% \renewcommand*{\Affilfont}{\normalfont\itshape}
\pretitle{\begin{center}\vskip -50pt}%
\title{\large \doctitle}
\posttitle{\end{center}}
\preauthor{\begin{center} \vskip -20pt}
\author{}
\postauthor{\end{center} \vskip -20pt}
\predate{\begin{center} \vskip -20pt}
\date{}%\small{\today}}
\postdate{\end{center} \vskip -20pt}% -42.5pt


% Fonts
\usepackage[singlespacing]{setspace}

 
% Math
\usepackage{amsmath}
\usepackage{amssymb}
\usepackage{physics}




 % Commands
\newcommand{\reals}{\mathbb{R}}
\DeclareMathOperator*{\argmax}{arg\,max}
\DeclareMathOperator*{\argmin}{arg\,min}





%%%%%%%%%%%%%%%%%%%%%%%%%%%%%%%%%%%%%%%%%%%%%%%%%%%
\begin{document}
\maketitle
\thispagestyle{fancy}
\singlespacing
%%%%%%%%%%%%%%%%%%%%%%%%%%%%%%%%%%%%%%%%%%%%%%%%

The major theme of research centres around methods from graph theory \cite{West2001,Newman2010}. A graph, denoted by $G = (V,E)$, consists of a set of vertices, $V$, with $|V| = n$. A single vertex is denoted by  $ i \in V$, and pairs of vertices are connected by a set of edges, $e=(i,j) \in E$, where $|E| = m$. The graph can also have local attributes on the vertices and edges, such as edge weights $w_{ij}$. In order to connect this abstract formalism with physics to be studied, states of a physical system are typically associated with vertices, and relationships between the states are described by the presence of edges \cite{Banerjee2019}.

Given this graph structure, rigorous analyses of graphs and the systems they represent using spectral analysis of various matrices that can be constructed based on the edges present in graphs. The eigenspaces of these matrices can then be used to describe the connectivity of graphs. The various graph matrices are summarized in Table \ref{tab:graphmatrices}, indicating their dimensions, relationship to the components of the graph, and how they relate to ideas of connectivity.
 
\begin{table}[htb!]
    \centering
    \caption{Summary of graph matrices and their associated connectivity measures.}
    \label{tab:graphmatrices}
    \begin{tabular}{|c|c|c|c|} \hline
        Matrix & Symbol & Dimensions &  Description            \\ \hline
        Adjacency & $A_{ij}$ & $n \times n$ & Number of edges between vertices $i$ and $j$. \\ \hline
        Degree & $D_{ij}$ & $n\times n$ & Number of neighbouring connections to other vertices at vertex $i$. \\ \hline
        Incidence & $M_{ij}$ & $n \times m$ & Whether a vertex $i$ is part of edge $j$. \\ \hline
        Laplacian & $L_{ij}$ & $n \times n$ & Smoothness of graph and connectivity. \\ \hline
    \end{tabular}
\end{table}

The adjacency matrix $A \in \reals^{n \times n}$, with elements $A_{ij} = |\{e : e=(i,j)\in E, j \neq i\}|$ corresponds to the number of edges between vertices $i,j$, and $<ij>$ generally denotes the nearest neighbours of $i$. This definition can also be extended to be the total weight of such edges in the case of weighted graphs. An important centrality measure for each vertex $i$ is the eigenvector centrality. Using the Perron-Frobenius theorem \cite{Newman2010}, this quantity is defined as the $i^{\textrm{th}}$ component of the eigenvector $x$ corresponding to the largest eigenvalue $\lambda_{\textrm{max}}$:
\begin{equation}
  Ax = \lambda_{\textrm{max}} x ~~\longleftrightarrow~~ x_i = \frac{1}{\lambda_{\textrm{max}}} \sum_{<ij>}x_j
\end{equation}

The diagonal degree matrix $D \in \reals^{n\times n}$, with elements $d_{i} = |\{e : e=(i,j)\in E, j \neq i\}|$ represents the degree, or number of edges adjacent to vertex $i$. Graphs where are all vertices have equal degree $d$ are denoted regular, and for directed graphs, the out-degree and in-degree are generally distinguished. The degree centrality measure is simply the normalized degree $\frac{d_i}{n-1}$, where a maximally connected simple clique graph has degree $n-1$. There are several formulas relating the degree and other graph quantities, including the degree sum formula 
\begin{align}
  \sum_i d_i ~=&~ 2|E| \label{eq:degreesum}, \intertext{and that the minimum $\delta$ and maximum $\Delta$ degree of a graph's vertices satisfies the bounds of} 
  \delta \leq 2 \frac{|E|}{|V|}&~ \leq \Delta.
\end{align}

The incidence matrix $M \in \reals^{n\times m}$, has non-zero elements for each pairing of vertex $i$ and edge $e_j$ when the vertex is a member of the edge: $i \in e_j$. $M_{ij} = \pm 1$, when $i$ is the outgoing or incoming vertex in the edge tuple: $e_j = (i,k) ~\textrm{or}~ (k,i), ~\textrm{for}~ k \in V$.

These matrices yield the most fascinating graph matrix, the Laplacian $L \in \reals^{n \times n}$, where
\begin{equation}
  L = D - A
\end{equation}
and in the case of undirected graphs, is symmetric and $L=MM^T$.

The graph Laplacian is a discrete equivalent of the Laplacian differential operator \cite{Chung1996}. This can most easily be seen by reasoning that for any vector $x \in \reals^n$, by definition of the adjacency and degree matrices being related to the total number of edges at a vertex, and knowing the degree sum form in Equation \ref{eq:degreesum}, the inner product of $x$ with respect to $L$ yields a second order finite difference formula:
\begin{equation}
  x^TLx = \sum_{e=(i,j)\in E} (x_i-x_j)^2.
\end{equation}
Therefore this positive semi-definite quadratic form for $x$ can be thought of as a measure of the vector's smoothness over the graph, similar to conclusions about continuous harmonic functions with respect to the differential Laplacian.

Among many interesting properties related to $L$, its eigenvalues yield information to assess the connectivity of the graph, the presence of any spanning subgraphs, possible random traversals of the graph, and notions of flow and conductance along edges \cite{Chung1996}. For clique graphs, the Laplacian takes the form $L = nI - ii^T$, where $i$ is the vector of all ones. The Laplacian matrix for the clique is not full rank, with one zero eigenvalue for eigenvector $\mathcal{i}$, and with $n-1$ constant non-zero eigenvalues of $\lambda = n$. The clique adjacency matrix has one eigenvalue of $n-1$, and $n-1$ eigenvalues of $-1$. The eigenvector centrality for cliques is therefore $1$ for each vertex, and this normalizes all other centrality measures when comparing the relative connectivity of graphs.

Given these various matrices and properties of their eigenspaces related to the associated graph connectivity, an important aspect of these graph theoretic approaches is finding these eigenvalues, or bounds on them using various approaches from numerical methods, including direct and iterative solvers.






%%%%%%%%%%%%%%%%%%%%%%%%%%%%%%%%%%%%%%%%%%%%%%%%
\newpage
\bibliographystyle{unsrt}
\bibliography{mduschen_Ex1.bib}

\end{document}